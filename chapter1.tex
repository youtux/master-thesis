% !TEX root = thesis.tex

\chapter{Background and related work}
\label{cha:background}
TODO


\section{Wikipedia}
\label{sec:wikistructure}
In this section we describe the various relevant entities that compose Wikipedia.

A Wikipedia \emph{article}, or entry, is a page that has encyclopedic information on it.
A well-written encyclopedia article identifies a notable encyclopedic topic, summarizes that topic comprehensively, contains references to reliable sources, and links to other related topics.

Articles belong to the main namespace of Wikipedia pages (also called ``article namespace'' or simply ``mainspace'').
The main namespace is the namespace of Wikipedia that contains the encyclopedia proper — that is, where ``live'' Wikipedia articles reside.
The main namespace is the default namespace and does not use a prefix in article page names.
This is distinct from other namespaces where page names are always prefixed by an indicator of the particular namespace in which the page resides.
For example, all user pages are prefixed by \textbf{User:}, their talk pages by \textbf{User talk:}, templates by \textbf{Template:} and various types of internal administrative pages by \textbf{Wikipedia:}.
Thus, any page created that lacks of such a prefix will automatically be placed under the article namespace.

\section{Microsoft academic graphs}
\label{sec:mag}


\section{Related work}
\label{sec:relatedwork}
