% !TEX root = thesis.tex

\chapter{Data analysis}
\label{cha:data_analysis}
The datasets created allow to perform some measurement regarding the use of scholarly works in the English Wikipedia.
\todo{TODO\@: other statistics? Maybe some that use the pagecounts?}

\section{Wikipedia articles}
\subsection{Where do identifiers appear in a Wikipedia article?}

\subsection{How many sections Wikipedia articles have?}

\subsection{Which are the most common sections in Wikipedia articles?}

\section{Papers in Wikipedia}
This section analyzes the use of papers in wikipedia.

\subsection{Incoming citations distribution}
The distribution of scholarly citations has been studied many times by the scientific community.
The problem is to discovery what is the distribution of citations, that is the number of papers cited a total number of $x$ times, $N(x)$.
Redner~\cite{Redner1998} suggested that this distribution has a large-$x$ power law decay $N(x) \sim x^{-\alpha}$.

Fig.~\ref{fig:incoming_citations_loglog} compares the distribution of citations of all the papers in history and the ones found only in Wikipedia.
The number of incoming citation for a paper is determined using the \ac{MAG} dataset.
Notice that because of the data available in the \ac{MAG}, only the papers found in Wikipedia which have a \ac{DOI} are considered.
This account for the 28\% of the identifiers found.
Furthermore, from this set of papers we removed the ones that appears to be ``non-valid'' in the \ac{MAG} dataset, namely the ones that have more than one \ac{DOI}.
The amount of papers considered at the end is 17\% of the identifiers found in wikipedia.

From the graph we can only see that both the distribution follow some kind of power law.
It is more interesting to analyze the \ac{CCDF} of the two series (Figures~\ref{fig:incoming_citations_ccdf_1000}~and~\ref{fig:incoming_citations_ccdf_100}).
In our case, the \ac{CCDF}, or \emph{tail distribution}, expresses the percentage of papers $\bar{F}(x)$ that have more than $x$ citations.
From the two figures we can see that papers appearing in Wikipedia tend to have many more incoming citations with respect to a random paper taken from all the available ones.
For instance, the 74\% of papers in Wikipedia have at least 10 incoming citations, while only the 13\% of papers have at least 10 incoming citations.
\todo{maybe limit to DOI only for the all papers distribution?}


\begin{figure}[ht]
    \centering
    \includegraphics[keepaspectratio=true, width=\textwidth]{assets/incoming_cits_loglog}
    \caption{Incoming citations distribution of papers appearing in the \emph{mag} datasets and the ones appearing only in English Wikipedia, on a log-log scale.}
    \label{fig:incoming_citations_loglog}
\end{figure}

\begin{figure}[ht]
    \centering
    \includegraphics[keepaspectratio=true, width=\textwidth]{assets/incoming_cits_ccdf_1000}
    \caption{Incoming citations distribution of papers appearing in the \emph{mag} datasets and the ones appearing only in English Wikipedia, on a log scale on the Y axis.}
    \label{fig:incoming_citations_ccdf_1000}
\end{figure}

\begin{figure}[ht]
    \centering
    \includegraphics[keepaspectratio=true, width=\textwidth]{assets/incoming_cits_ccdf_100}
    \caption{Incoming citations distribution of papers appearing in the \emph{mag} datasets and the ones appearing only in English Wikipedia, on a log scale on the Y axis.}
    \label{fig:incoming_citations_ccdf_100}
\end{figure}



\subsection{Publication date distribution}
see https://en.wikipedia.org/wiki/User:ProteinBoxBot.

todo: show graph of identifier insertions in time

analizza la persistenza degli identifier inseriti e rimossi

fare la average della persistenza degli identifier escludendo la coda

\subsection{Age at first appearance}

\subsection{Lifetime in a Wikipedia article}

\subsection{Top cited journals by views}
